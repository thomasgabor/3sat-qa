\section{Introduction}
%Following the great paper \cite{feld2018hybrid} we thought, let's do another one!
%\todo{0.5 Introduction}

The benefit of the technology of quantum computing essentially lies in the quantum speedup. More generally, if quantum computing is to succeed, it needs to be faster or better or cheaper than classical computing hardware at at least one useful task.

Research in that area has cast an eye on the complexity class NP: It contains problems that are traditionally (and at the current state of knowledge regarding the P vs. NP problem) conjectured to produce instances too hard for classical computers to solve within practical time constraints. However, the problems in NP are also known to be quite susceptible to the benefits of parallelization.

As quantum mechanics has an interpretation suggesting infinite parallelism of possibilities contained within a superposition, we may have a match here. For a rather concrete implementation of quantum mechanics we focus on the technological platform of quantum annealing. Quantum annealers focus on solving optimization problems only. As a trade-off they can as of now work rather large amounts of qubits in a useful way and are one of the few technological platforms in the area of quantum technology that are easily available to the general public.

In this paper we want to evaluate the performance of quantum annealing (or more specifically a D-Wave 2000Q machine) on the canonical problem of the class NP, i.e., propositional logic satisfiability for 3-literal clauses (3SAT) \cite{cook1971complexity}. As we note that there is still a remarkable gap between 3SAT instances that can be put on a current D-Wave chip and 3SAT instances that even remotely pose a challenge to classical solvers, there is little sense in comparing the quantum annealing method to classical algorithms in this case. Instead, we are most interested in the scaling behavior with respect to problem difficulty. Or more precisely: We analyze if and to what extent quantum annealing's performance suffers under hard problem instances (like classical algorithms do).

To this end, we present a quick run-down of 3SAT and the phenomenon of phase transitions in Section~\ref{sec:preliminaries} and continue to discuss further related work in Section~\ref{sec:related-work}. In Section~\ref{sec:exp-setup} we describe our experimental setup and then present the corresponding results in Section~\ref{sec:evaluation}. We conclude with Section~\ref{sec:conclusion}.
