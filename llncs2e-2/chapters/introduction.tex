\section{Introduction}
%Following the great paper \cite{feld2018hybrid} we thought, let's do another one!
%\todo{0.5 Introduction}

%The benefit of the technology of quantum computing is essentially connected to achieving quantum speedups. Or, phrased more generally, if quantum computing is to succeed, it needs to be faster or better or cheaper than classical computing hardware on at least one useful task.

Quantum computers are an emerging technology and still subject to frequent new developments. Eventually, the utilization of intricate physical phenomena like superposition and entanglement is conjectured to provide an advantage in computational power over purely classical computers. As of now, however, the first practical breakthrough application for quantum computers is still sought for. But new results on the behavior of quantum programs in comparison to their classical counterparts are reported on a daily basis.

Research in that area has cast an eye on the complexity class NP: It contains problems that are traditionally (and at the current state of knowledge regarding the P vs.\ NP problem) conjectured to produce instances too hard for classical computers to solve exactly and deterministically within practical time constraints. Still, problem instances of NP are also easy enough that they can be executed efficiently on a (hypothetical) non-deterministic computer.

The notion of computational complexity is based on classical
computation in the sense of using classical mechanics to describe and
perform automated computations. In particular, it is known that in this
model of computation, simulating quantum mechanical systems is hard. However, nature itself routinely ``executes'' quantum mechanics, leading to
speculations~\cite{feynman1981simulating} that quantum mechanics may be used to leverage 
 greater computational power than systems adhering to the rules of classical physics
can provide.

Quantum computing describes technology exploiting the behavior of quantum mechanics to build computers that are (hopefully) more powerful than current classical machines. Instead of classical bits $b \in \{0, 1\}$ they use qubits $q = \alpha\ket{0} + \beta\ket{1}$ where $\alpha, \beta, |\alpha|^2 + |\beta|^2 = 1,$ are probability amplitudes for the basis states $\ket{0}, \ket{1}$. Essentially, a qubit can be in both states $0$ and $1$ at once. This phenomenon is called superposition, but it collapses when the actual value of the qubit is measured, returning either $0$ or $1$ with a specific probability and fixing that randomly acquired result as the future state of the qubit. Entanglement describes the effect that multiple qubits can be in superpositions that are affected by each other, meaning that the measurement of one qubit can change the assigned probability amplitudes of another qubit in superposition. The combination of these phenomena allows qubits to concisely represent complex data and lend themselves to efficient computation operations.

The technological platform of quantum annealing is (unlike the generalized concept of quantum computing) not capable of executing general quantum-mechanical computations, but is within current technological feasibility and available to researchers outside the field of quantum hardware. The mechanism specializes in solving optimization problems and can (as a trade-off) work larger amounts of qubits in a useful way than current quantum-mechanically complete platforms.

In this paper, we evaluate the performance of quantum annealing (or more specifically, a D-Wave 2000Q machine) on the canonical problem of the class NP, propositional logic satisfiability for 3-literal clauses (3SAT)~\cite{cook1971complexity}. As we note that there is still a remarkable gap between 3SAT instances that can be put on a current D-Wave chip and 3SAT instances that even remotely pose a challenge to classical solvers, there is little sense in comparing the quantum annealing method to classical algorithms in this case (and at this early point in time for the development of quantum hardware). Instead, we are interested in the scaling behavior with respect to problem difficulty. Or more precisely: We analyze if and to what extent quantum annealing's performance suffers under hard problem instances (like classical algorithms do).

We present a quick run-down of 3SAT and the phenomenon of phase transitions in Section~\ref{sec:preliminaries} and continue to discuss further related work in Section~\ref{sec:related-work}. In Section~\ref{sec:exp-setup} we describe our experimental setup and then present the corresponding results in Section~\ref{sec:evaluation}. We conclude with Section~\ref{sec:conclusion}.
