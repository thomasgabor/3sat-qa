\section{Conclusion}
\label{sec:conclusion}

%\todo{1 Discussion \& Conclusion (7.5.3)}

We have shown that problem difficulty of 3SAT instances also affects the performance of quantum annealing as it does for classical algorithms. However, bound by the nature of both approaches, the effects are quite different with complete classical algorithms showing longer runtimes and quantum annealing showing less precision. A first quantification of that loss of precision suggests that it may not be too detrimental and comparatively easy to deal with. However, due to the maximum available chip size for quantum annealing hardware at the moment., no large-scale test could be performed. Thus, no real assumptions on the scaling of this phenomenon (and thus the eventual real-world benefit) can be made yet.

Our results suggest that there are cases where single solutions from a set of equally optimal solutions are much more likely to be returned than others. This observation is in line with other literature on the results of quantum annealing, however, it is interesting to note that it also translates into the original problem space of 3SAT.

The observed results gain practical relevance with larger chip sizes for quantum annealers. We thus suggest to perform these and/or similar tests for future editions of quantum annealing hardware. if the effects persist, they represent a huge advantage of quantum hardware over any other known approaches for solving NP-hard problems.
