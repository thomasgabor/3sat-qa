
\section{Related Work}
It is one of the cornerstones of complexity theory that solving NP-complete or even NP-hard decision problems is not efficiently possible (e.g., \cite{cook1971complexity,murty1987some}). Any NP-complete problem can also be cast as an optimisation problem, which allows for employing well-known optimisation algorithms to find approximate solutions -- typical methods include tabu search (see, e.g.,~\cite{glover2013tabu}, \cite{gendreau1994tabu}) and simulated annealing (see, e.g.,~\cite{kirkpatrick1983optimization}, \cite{chen1995chaotic}). Countless other efficient approximation methods, together with an elaborate taxonomy on approximation quality (what is the solution's distance from a known global optimum?) and computational effort (how many time steps are required until an approximate solution that satisfies given quality goals is available?), have been devised (see, e.g.,~\cite{NPO-Compendium}).

Some problems, for instance knapsack, exhibit favourable properties
when cast as an optimisation problem. The latter is a member of the
complexity class FPTAS (fully polynomial-time approximation scheme),
which means that a solution with distance \((1+\epsilon)\) ($\epsilon > 0$)
from an optimal solution can be determined in polynomial time in both,
input size \(n\) and \(1/\epsilon\)~\cite{chen1995chaotic}.

An intriguing connection that has received substantial attraction exists between (computational) NP-complete problems and the (physical) concept of phase transitions, as detailled in Sec.~\ref{TODO:Preliminaries}. First investigations of the phenomenon have been performed by Kirkpatrick et al.~\cite{kirkpatrick1994critical}, TODO~\cite{TODO monasson1999determining} and TODO~\cite{TODO}; more recent investigations include~\cite{TODO, TODO, TODO}.

The idea of obtaining solutions for NPO (optimisation) problems by finding the energy ground state(s) of a quantum mechanical system appeared at around 1988~\cite{TODO: Originalpaper,albash2016adiabatic}. Ausgangspunkt dieser Art kombinatorische Optimierungsprobleme zu lösen waren Apolloni et al. (vgl. \cite{apolloni1989quantum}, \cite{apolloni1988numerical}). The idea of quantum annealing has been independently re-discovered multiple times, see~\cite{albash2016adiabatic,finnila1994quantum,amara1993global,kadowaki1998quantum}).

Seit der ersten Erfindung des Quantum Annealings wurde für viele NP-vollständige Probleme eine Problemformulierung gefunden, so dass diese Probleme auf einem Quantenannealer gelöst werden können (vgl. \cite{lucas2014ising}) und bis heute findet viel Forschung für Probleme wie das Problem des Handlungsreisenden oder das 3SAT-Problem im Kontext des Quantencomputings statt (vgl. \cite{heim2017designing}, \cite{warren2017small}, \cite{moylett2017quantum}, \cite{strand2017zzz}, \cite{benjamin2017measurement}). Neben dem Quantum Annealing gibt es ein weiteres Model des Quantencomputings, genannt Quantum-Gate-Computing, welches polynomiell Äquivalent zum Model des Quantum Annealings ist \cite{mcgeoch2014adiabatic}, welches in dieser Arbeit jedoch nicht berücksichtigt wird. Mit der Erfindung des Quantencomputings, einer neunen Herangehensweise an das Lösen von Problemen, entstanden auch neue Komplexitätsklassen (vgl. \cite{klauck2017complexity}, \cite{morimae2017merlinization}), deren Beziehungen zur klassischen Komplexitätstheorie ebenfalls Teil der Forschung war (vgl. \cite{bernstein1997quantum}, \cite{marriott2005quantum}).\\\\Zu den in dieser Arbeit im Kontext des Quantencomputing betrachteten 3SAT-Problemen fand bereits viel Foschung statt. So wurden verschiedene Methoden vorgestellt, diese Probleme auf einem Quantenannealer zu lösen (vgl \cite{choi2011different}, \cite{choi2010adiabatic}, \cite{farhi2000quantum}). Ebenso fanden Untersuchungen zur Lösungskomplexität von 3SAT-Problemen im Kontext des Quantencomputings statt (vgl. \cite{van2001powerful}, \cite{farhi2000numerical}) und für bestimmte 3SAT-Instanzen die nicht effizient von einfachen Quantenalgorithmen gelöst werden können wurde eine mögliche Lösungsmehtode vorgeschlagen (vgl. \cite{farhi2009quantum}. Ebenso wurden Variationen des 3SAT-Problems, wie zum Beispiel das Exact Cover-Problem angewandt auf 3SAT-Formeln, betrachtet (vgl. \cite{farhi2001quantum}).

