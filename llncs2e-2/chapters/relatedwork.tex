\section{Related Work}
\label{sec:related-work}

It is one of the cornerstones of complexity theory that solving NP-complete or even NP-hard decision problems is strongly believed to be not efficiently possible (see, e.g.,~\cite{cook1971complexity,murty1987some}). Any NP-complete problem can also be cast as an optimization problem, which allows for employing well-known optimization algorithms to find approximate solutions---typical methods include tabu search (see, e.g.,~\cite{glover2013tabu,gendreau1994tabu}) and simulated annealing (see, e.g.,~\cite{kirkpatrick1983optimization,chen1995chaotic}). Countless other efficient approximation methods, together with an elaborate taxonomy on approximation quality (how much does a given solution differ from a known global optimum?) and computational effort (how many time steps are required until an approximate solution that satisfies given quality goals is available?), have been devised (see, e.g.,~\cite{Ausiello1999}).

Some problems, for instance knapsack, exhibit favorable properties when cast as an optimization problem. The latter is a member of the complexity class FPTAS (fully polynomial-time approximation scheme), which means that a solution with distance \(1+\epsilon\) (of course, $\epsilon > 0$) from an optimal solution can be determined in polynomial time in both, input size \(n\) and inverse approximation quality \(1/\epsilon\)~\cite{chen1995chaotic}.

An intriguing connection that has received substantial attraction exists between (computational) NP-complete problems and the (physical) concept of phase transitions, as detailed in Section~\ref{sec:preliminaries}. First investigations of the phenomenon have been performed by Kirkpatrick et al.~\cite{kirkpatrick1994critical}; Monasson et al.\ first suggested a connection between the type of phase transition and the associated computational costs of a problem~\cite{monasson1999determining}. From the abundant amount of more recent investigations, we would like highlight the proof by Ding et al.~\cite{ding2015proof} that establishes a threshold value for the phase transition. Our work benefits from the above insights by selecting the ``most interesting'', that is, computationally hardest scenarios as investigation target.

The idea of obtaining solutions for NPO (NP optimization) problems by finding the energy ground state (or states) of a quantum mechanical system was used, for instance, by Apolloni et al.~\cite{apolloni1989quantum,apolloni1988numerical} to solve combinatorial optimization problems. The general idea of quantum annealing has been independently re-discovered multiple times~\cite{albash2016adiabatic,finnila1994quantum,amara1993global,kadowaki1998quantum}.

Quantum annealing techniques are usually applied to solving NP-complete or NP-hard decision problems, or optimization problems from class NPO. Lucas~\cite{lucas2014ising} reviews how to formulate a set of key NP problems in the language of adiabatic quantum computing respectively quadratic unconstrained binary optimization (QUBO). In particular, problems of the types ``travelling salesman'' or ``binary satisfiability'' that are expected to have a major impact on practical computational applications if they can be solved advantageously on quantum annealers have undergone a considerable amount of research~\cite{heim2017designing,warren2017small,moylett2017quantum,strand2017zzz,benjamin2017measurement}.

Comparing the computational capabilities of classical and quantum computers is an intriguing and complex task, since the deployed resources are typically very dissimilar. For instance, the amount of instructions required to execute a particular algorithm is one of the main measures of efficiency or practicability on a classical machine, whereas the notion of a discrete computational ``step'' is hard to define on a quantum annealing device. Interest in quantum computing has also spawned definitions of new complexity classes (e.g., \cite{klauck2017complexity,morimae2017merlinization}), whose relations to traditional complexity classes have been and are still subject to ongoing research (see, e.g., \cite{bernstein1997quantum,marriott2005quantum}).

% TODO: Hier brauchen wir in der endgueltigen Version auf jeden Fall
% noch bessere Referenzen.

These questions hold regardless of any specific physical or conceptual implementation of quantum computing since their overall computational capabilities are known to be largely interchangeable; see, for instance, Reference~\cite{mcgeoch2014adiabatic} for a discussion on the
equivalence of gate-based and adiabatic quantum computing. Consequently,
our work focuses not on comparing quantum and classical aspects of solving particular problems, but concentrates on understanding peculiarities
of solving one particular problem (3SAT, in our case) in-depth.

Formulating 3SAT problems on a quantum annealing hardware has been previously considered~\cite{choi2011different,choi2010adiabatic,farhi2000quantum}, and we rely on the encoding techniques presented there. Van~\cite{van2001powerful} and Farhi~\cite{farhi2009quantum} have worked on analyzing the complexity of solving general 3SAT problems.
